\documentclass{beamer}

\usepackage{../macros}

\title{Input / Output}

\begin{document}

\frame{
  \titlepage
}

\section{Examples}

\begin{frame}[fragile]{Examples of Input/Output}

  \begin{code}{In Python}
    \begin{Python}
n = input('Type an integer\n')
print('You entered {}\n'.format(n))
    \end{Python}
  \end{code}

  \begin{code}{In C}
    \begin{C}
printf("Type an integer\n");
int n;
scanf("%d", &n);
printf("You entered %d\n", n);
    \end{C}
  \end{code}

  \begin{code}{In Java}
    \begin{Java}
Scanner sc = new Scanner (System.in);
System.out.println("Type an integer");
int n = sc.nextInt();
System.out.println("You entered " + n);
    \end{Java}
  \end{code}

\end{frame}

\begin{frame}[fragile]{Examples of Input/Output}

  \begin{code}{In R}
    \begin{R}
n <- readline(promtp("Type an integer")
# conversion in integer
n <- as.integer(n)
print(paste("You entered", n))
    \end{R}
  \end{code}

  \begin{code}{In shell}
    \begin{Shell}
echo "Type an integer"
read n
echo "You entered $n"
    \end{Shell}
  \end{code}

\end{frame}


\section{Exercices}

\end{document}
