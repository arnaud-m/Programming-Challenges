
\section{Input / Output}

\begin{frame}{}
  \Large
  \centering
  \color{bleu}
  Input / Output
\end{frame}

\subsection{Examples}

\begin{frame}[fragile]{Examples of Input/Output}
  
  \begin{block}{In Python}
    \begin{lstlisting}[style=codePy]
n = input('Type an integer\n')
print('You entered {}\n'.format(n))
    \end{lstlisting}
  \end{block}
  
  \begin{block}{In C}
    \begin{lstlisting}[style=codeC]
printf("Type an integer\n");
int n;
scanf("%d", &n);
printf("You entered %d\n", n);
    \end{lstlisting}
  \end{block}
  
  \begin{block}{In Java}
    \begin{lstlisting}[style=codeJ]
Scanner sc = new Scanner (System.in);
System.out.println("Type an integer");
int n = sc.nextInt();
System.out.println("You entered " + n);
    \end{lstlisting}
  \end{block}
  
\end{frame}

\begin{frame}[fragile]{Examples of Input/Output}
  
  \begin{block}{In R}
    \begin{lstlisting}[style=codeR]
n <- readline(promtp("Type an integer")
# conversion in integer
n <- as.integer(n)
print(paste("You entered", n))
    \end{lstlisting}
  \end{block}
  
  \begin{block}{In shell}
    \begin{lstlisting}[style=codeS]
echo "Type an integer"
read n
echo "You entered $n"
    \end{lstlisting}
  \end{block}
  
\end{frame}


\subsection{Exercices}