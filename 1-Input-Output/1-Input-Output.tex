\documentclass{beamer}

\usepackage{../macros}
\usepackage[nofillcomment,vlined,ruled]{algorithm2e}
\SetAlFnt{\small}
\SetAlCapFnt{\large}
\SetAlCapNameFnt{\large}

\title{Input / Output}

\begin{document}

\frame{
  \titlepage
}


\section{General Methods for Input/Output}



\begin{frame}[fragile]{Methods of Input/Output}
 There are several approaches to reading in the text input.
 \begin{itemize}
 \item Repeatedly get single characters
   \begin{itemize}
   \item (perhaps using a library function like getchar);
   \end{itemize}
 \item Repeatedly get strings and break them down into single characters.
   \begin{itemize}
   \item (perhaps using a library function like scanf). 
   \end{itemize}
 \item Read the entire line as a string, and then parsing it by accessing characters in the string.
   \begin{itemize}
   \item (perhaps using a library function like gets).
   \end{itemize}
\end{itemize}
Perhaps more modern ways using streams are easier, perhaps not.
\end{frame}


\begin{frame}{Make the Best from your Language}
  
  \begin{block}{Basic Data Types}
    Selecting the right data structure makes a trememdous difference in the organization and complexity of a given program.
    Be aware of your basic structured data types (arrays, records, multidimensional arrays, enumerated types) and what they are used for.
\end{block}

\begin{block}{Library Functions}
  
\end{block}
\end{frame}


\section{Character Codes}
\begin{frame}{Character Codes}
  \begin{alertblock}{Character codes}
    Character codes are mappings between numbers and the symbols which make up a particular alphabet.
  \end{alertblock}
  \begin{block}{ASCII}
    The American Standard Code for Information Interchange (ASCII) is a single-byte character code where $2^7 = 128$ characters are specified.
    Bytes are eight-bit entities; so that means the highest-order bit is left as zero.
    
  \end{block}
  

\end{frame}

\begin{frame}{Properties of ASCII}
  \begin{itemize}
  %\item All non-printable characters have either the first three bits as zero or all seven lowest bits as one. This makes it very easy to eliminate them before displaying junk.
  \item Both the uppercase and lowercase letters and the numerical digits appear sequentially. Thus we can iterate through all the letters/digits simply by looping from the value of the first symbol (say, 'a') to value of the last symbol (say, 'z'). 
  \item We can convert a character (say, 'I') to its rank in the collating sequence (eighth, if 'A' is the zeroth character) simply by subtracting off the first symbol ('A').
  \item We can convert (say 'C') from uppercase to lowercase by adding the difference of the upper and lowercase starting character ('C'-'A'+'a')
  \item The character code tells us what will happen when naively sorting text files. Sorting alphabetically means sorting by character code. Using a different collating sequence requires more complicated comparison functions.
  \item Non-printable character codes for new-line (10) and carriage return (13) are designed to delimit the end of text lines.
    Inconsistent use of these codes is one of the pains in moving text files between UNIX and Windows systems.
 \end{itemize}
\end{frame}


\begin{frame}{Unicode}
  More modern international character code designs such as Unicode use two or even three bytes per symbol, and can represent virtually any symbol in every language on earth.
  \begin{itemize}
  \item Older languages, like Pascal, C, and C++, view the char type as virtually synonymous with 8-bit entities.
  \item However, good old ASCII remains alive embedded in Unicode. Java, on the other hand, was designed to support Unicode, so characters are 16-bit entities. 
  \end{itemize}


\end{frame}

\section{String Representation}


\begin{frame}{Representing Strings}
 Strings are sequences of characters, where order clearly matters.
 It is important to be aware of how your favorite programming language represents strings, because there are several different possibilities:
 \begin{description}
 \item[Null-terminated Arrays] C/C++ treats strings as arrays of characters.
  The string ends the instant it hits the null character '\\0', i.e., zero ASCII.
  Failing to end your string explicitly with a null typically extends it by a bunch of unprintable characters
 \item[Array Plus Length] Another scheme uses the first array location to store the length of the string, thus avoiding the need for any terminating null character.
  Presumably this is what Java implementations do internally.
 \item[Linked Lists of Characters] Text strings can be represented using linked lists, but this is typically avoided because of the high space-overhead associated with having a several-byte pointer for each single byte character.
 \end{description}
 

\end{frame}

\begin{frame}{Which String Representation?}
The underlying string representation can have a big impact
on which operations are easily or efficiently supported.
Compare each of these three data structures with respect to
the following properties:

\begin{itemize}
\item Which uses the least amount of space? On what sized strings?
\item Which constrains the content of the strings which can possibly be represented?
\item Which allow constant-time access to the i-th character?
\item Which allow efficient checks that the i-th character is in fact within the string, thus avoiding out-of-bounds errors?
\item Which allow efficient deletion or insertion of new characters at the i-th position?
\end{itemize}

\end{frame}


% \begin{frame}{String Library Functions}

%   \begin{block}{C String Library Functions}
%     The C language character library ctype.h contains several simple tests and manipulations on character codes.
%     As with all C predicates, true is defined as any non-zero quantity, and false as zero.
%     The C language string library is string.h.
%   \end{block}

% \begin{block}{C++ String Library Functions}
% In addition to supporting C-style strings, C++ has a string class which contains methods for these operations and more. Overloaded operators exist for concatenation and string comparison.
% \end{block}
% \begin{block}{java String Objects}
%   Java strings are first-class objects deriving either from the String class or the Stringuffer/StringBuilder class.
%   The String class is for static strings which do not change, while others are designed for dynamic strings.
%   Recall that Java was designed to support Unicode, so its characters are 16-bit entities.
%   The java.text package contains more advanced operations on strings, including routines to parse dates and other structured text.
% \end{block}

% \end{frame}


\section{I/O Bounds}

\begin{frame}{I/O Bounds}

  \begin{block}{Definition}
    The I/O bound refers to a condition in which the time it takes to complete a computation is determined principally by the period spent waiting for input/output operations to be completed.
  \end{block}

  \begin{block}{Van Neuman Bottleneck}
    Since data must be moved between the CPU and memory along a bus which has a limited data transfer rate, the data bandwidth between the CPU and memory tends to limit the overall speed of computation.
  \end{block}

  \begin{block}{Why is it undesirable?}
    The CPU must stall its operation while waiting for data to be loaded or unloaded from main memory or secondary storage.
  \end{block}
  
\end{frame}


\section{Hints for the Problems}

\begin{frame}
  \frametitle{TEST - Life, the Universe, and Everything}

  \begin{alertblock}{Quickstart}
    Simply modify a solution of the \href{https://github.com/arnaud-m/echo}{ECHO Problem} in your preferred language.
  \end{alertblock}
  
  \begin{block}{\href{https://github.com/arnaud-m/echo}{ECHO Problem}}
    The \href{https://github.com/arnaud-m/echo}{ECHO repository} contains templates for some popular languages. You can contribute if your favorite language is missing.
  \end{block}
\end{frame}


\begin{frame}[fragile]
  \frametitle{HELLOKIT - Hello Kitty}

  \begin{algorithm}[H]
    \KwData{a word $w$ and a positive integer $n$.}
    \SetKwFunction{print}{print}
    \For{$i \in |w|\ldots1$}{
      \print the last $i$ letters of $w$\;
      \print $n-1$ times the word $w$\;
      \print the first $i-1$ letters of $w$\;
    }
    \caption{RectangularPattern}
  \end{algorithm}

  \begin{columns}[t]
    \begin{column}{0.49\textwidth}
      \begin{block}{With String Arithmetic in Python}
        \begin{Python}
>>> x = 'Love'
>>> x[-1] + x[0:len(x)-1]
'eLov'
\end{Python}
\end{block}
    \end{column}
    
    \begin{column}{0.49\textwidth}
      \begin{block}{With Character Buffer in C}
        \begin{C}
char w[] = "Love";
printf("%s%.2s", &v[2], v);
// prints 'veLo'
        \end{C}
      \end{block}
    \end{column}
  \end{columns}
\end{frame}


\begin{frame}[fragile]
  \frametitle{706 - LC-Display}
  
    \begin{algorithm}[H]
      \SetKwFunction{print}{print}
      \KwData{an integer $n$ and a size $w$.}
      \ForEach{$r \in [1, 2s+3]$}{
        \tcc{In practice, split the loop into multiple ones}
        \ForEach{digit $d$ of $n$}{
          \print the $r$-th row with size $s$ of the digit $d$\;
          \print a space if $d$ is not the last digit of $n$
        }
        \print a newline
      }
      \caption{LC-Display}
  \end{algorithm}

  \begin{block}{Encoding a Digit of size 1 }
    \begin{C}
const char lookupTable[10][5][3] = {
  { " - ",
    "| |",
    "   ",
    "| |",
    " - "},
    ...};
\end{C}
  \end{block}
\end{frame}
\end{document}


%%% Local Variables:
%%% mode: latex
%%% TeX-master: t
%%% End:
